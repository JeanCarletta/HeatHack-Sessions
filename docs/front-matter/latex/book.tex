%% Generated by Sphinx.
\def\sphinxdocclass{jupyterBook}
\documentclass[letterpaper,10pt,english]{jupyterBook}
\ifdefined\pdfpxdimen
   \let\sphinxpxdimen\pdfpxdimen\else\newdimen\sphinxpxdimen
\fi \sphinxpxdimen=.75bp\relax
\ifdefined\pdfimageresolution
    \pdfimageresolution= \numexpr \dimexpr1in\relax/\sphinxpxdimen\relax
\fi
%% let collapsible pdf bookmarks panel have high depth per default
\PassOptionsToPackage{bookmarksdepth=5}{hyperref}
%% turn off hyperref patch of \index as sphinx.xdy xindy module takes care of
%% suitable \hyperpage mark-up, working around hyperref-xindy incompatibility
\PassOptionsToPackage{hyperindex=false}{hyperref}
%% memoir class requires extra handling
\makeatletter\@ifclassloaded{memoir}
{\ifdefined\memhyperindexfalse\memhyperindexfalse\fi}{}\makeatother

\PassOptionsToPackage{warn}{textcomp}

\catcode`^^^^00a0\active\protected\def^^^^00a0{\leavevmode\nobreak\ }
\usepackage{cmap}
\usepackage{fontspec}
\defaultfontfeatures[\rmfamily,\sffamily,\ttfamily]{}
\usepackage{amsmath,amssymb,amstext}
\usepackage{polyglossia}
\setmainlanguage{english}



\setmainfont{FreeSerif}[
  Extension      = .otf,
  UprightFont    = *,
  ItalicFont     = *Italic,
  BoldFont       = *Bold,
  BoldItalicFont = *BoldItalic
]
\setsansfont{FreeSans}[
  Extension      = .otf,
  UprightFont    = *,
  ItalicFont     = *Oblique,
  BoldFont       = *Bold,
  BoldItalicFont = *BoldOblique,
]
\setmonofont{FreeMono}[
  Extension      = .otf,
  UprightFont    = *,
  ItalicFont     = *Oblique,
  BoldFont       = *Bold,
  BoldItalicFont = *BoldOblique,
]



\usepackage[Bjarne]{fncychap}
\usepackage[,numfigreset=1,mathnumfig]{sphinx}

\fvset{fontsize=\small}
\usepackage{geometry}


% Include hyperref last.
\usepackage{hyperref}
% Fix anchor placement for figures with captions.
\usepackage{hypcap}% it must be loaded after hyperref.
% Set up styles of URL: it should be placed after hyperref.
\urlstyle{same}


\usepackage{sphinxmessages}



        % Start of preamble defined in sphinx-jupyterbook-latex %
         \usepackage[Latin,Greek]{ucharclasses}
        \usepackage{unicode-math}
        % fixing title of the toc
        \addto\captionsenglish{\renewcommand{\contentsname}{Contents}}
        \hypersetup{
            pdfencoding=auto,
            psdextra
        }
        % End of preamble defined in sphinx-jupyterbook-latex %
        

\title{Introducing HeatHack}
\date{Mar 03, 2022}
\release{}
\author{Jean Carletta}
\newcommand{\sphinxlogo}{\vbox{}}
\renewcommand{\releasename}{}
\makeindex
\begin{document}

\pagestyle{empty}
\sphinxmaketitle
\pagestyle{plain}
\sphinxtableofcontents
\pagestyle{normal}
\phantomsection\label{\detokenize{intro::doc}}


\sphinxAtStartPar
Knowing how to manage energy use in church buildings and halls is difficult.  Many of us manage the oldest and most decorative premises in our communities.  We are, for the most part, volunteers who know our buildings well, but have no training that helps us understand how the energy systems in our buildings work and how to change them for the better.  Meanwhile, there are few professionals who spend enough of their working lives thinking about churches to provide us with confident advice.  This means sometimes we are told things that are more appropriate for houses, offices, schools, and factories.  Even when the professionals have time to think deeply about our needs, sometimes the equipment then can buy is designed with very different buildings in mind.  It’s understandable, then, that we might not be getting the best service from the energy we put into our buildings.

\sphinxAtStartPar
Now that energy is expensive and we understand that we need to move to net zero to reduce the harmful impacts of climate change, it is more important than ever not to waste energy.  Getting this right will help our buildings and the services provided within them continue to survive, and we hope, even to thrive.

\sphinxAtStartPar
This programme is designed to help you understand energy use in your premises and what net zero means for you.  We hope you find the programme useful in your journey, and that together we can build a better future for all of us.
\begin{itemize}
\item {} 
\sphinxAtStartPar
{\hyperref[\detokenize{dedication::doc}]{\sphinxcrossref{Dedication}}}

\item {} 
\sphinxAtStartPar
{\hyperref[\detokenize{motivation::doc}]{\sphinxcrossref{Why take part?}}}

\item {} 
\sphinxAtStartPar
{\hyperref[\detokenize{learning::doc}]{\sphinxcrossref{What you will learn}}}

\item {} 
\sphinxAtStartPar
{\hyperref[\detokenize{why-engineering::doc}]{\sphinxcrossref{Why Engineering?}}}

\item {} 
\sphinxAtStartPar
{\hyperref[\detokenize{about::doc}]{\sphinxcrossref{About this programme}}}

\end{itemize}


\chapter{Dedication}
\label{\detokenize{dedication:dedication}}\label{\detokenize{dedication::doc}}
\sphinxAtStartPar
This programme is dedicated to the Boston Women’s Health Collective.  Since the 1970s, they have been empowering women to collaborate in their own health care.  This is oddly similar to what we want to achieve!  It is also dedicated to volunteers everywhere helping to look after our community buildings.


\chapter{Why take part?}
\label{\detokenize{motivation:why-take-part}}\label{\detokenize{motivation::doc}}
\sphinxAtStartPar
This programme assumes your group needs to understand what a net zero future means for your premises.  For some groups, this need will arise from a deeply felt ethical concern for the planet and its inhabitants.  For others, it arises because of pressure from above, as church hierarchies make policies requiring net zero action plans.  For some, volatile and rising energy prices will be foremost in their minds, or the dwindling resources of an aging congregation.  All of these are perfectly valid reasons for taking part.

\begin{sphinxShadowBox}
\sphinxstylesidebartitle{Further reading}
\begin{itemize}
\item {} 
\sphinxAtStartPar
Carbon Conversations.  \sphinxurl{http://www.carbonconversations.co.uk/}

\item {} 
\sphinxAtStartPar
Climate Conversations. \sphinxurl{https://www.climateconversations.net/}

\end{itemize}
\end{sphinxShadowBox}

\sphinxAtStartPar
This programme is not the place to debate whether climate change exists or whether human behaviour has the power to affect its course for good or ill.  It is also not the place to have detailed discussions about climate change and its effects on the planet, although if that is your main interest, you may wish to consider our sister programme, Climate Conversations.  Our programme is about how to get the most benefit out of the energy you put into your buildings and how to have what are difficult discussions about whether that energy use can be supported financially and justified ethically in your place and at this time.   It is based loosely on our experiences in running Carbon Conversations, structured discussion groups about how to achieve reductions in the carbon impacts of our personal lives.


\chapter{What you will learn}
\label{\detokenize{learning:what-you-will-learn}}\label{\detokenize{learning::doc}}
\sphinxAtStartPar
In the programme, you will learn how heating and ventilation systems work and where the big heat losses in traditional buildings occur. You will use that learning to understand your own buildings.  You will also learn about thermal comfort, where the temperature a thermostat shows is not the whole story, and what this means for how you use your spaces. Finally, you will learn how to check whether your systems are working correctly, fit for purpose, and how to plan for future changes.  This might sound technical and scary, but don’t worry, we aren’t trying to make you into an engineer, just help you know what they do and work with them better!

\sphinxAtStartPar
To help you understand your buildings, we will provide you with little devices you can put in your spaces that will collect regular temperature and relative humidity readings.  You do not need to know how they work, but with the aid of your group’s engineer, you will learn how to spot some basic problems on the data they provide.  We also have some specialist equipment available by post for those who need it to look at specific issues.  The data our equipment collects, along with other information you will be collecting, will form a profile of your premises and how you use them.  This profile is intended to help professionals understand what you need and serve you better.

\sphinxAtStartPar
Much of the programme isn’t technical at all \sphinxhyphen{} it’s discussion about what your building is for and how it should be used and managed in the future.  For this, towards the end, you’ll need to engage with others in your local community.  After all, it’s space that serves them.  This kind of conversation will be a new one to many groups, and so the programme is designed to equip you for that, too, by showing you some good ways to engage the people you serve.


\chapter{Why Engineering?}
\label{\detokenize{why-engineering:why-engineering}}\label{\detokenize{why-engineering::doc}}
\sphinxAtStartPar
This programme is unusual for ones run in churches, because it introduces what may be a new kind of specialist to your group: an engineer.  Engineers use science to solve problems.  They’re also good at thinking about time and budget constraints, because the whole point of engineering is to make things that work in the real world. It can be useful to have an engineer involved when you’re thinking about your property just because of the way they are trained to think about problems and potential solutions.  You should come away from the programme with a new appreciation of engineering and, we hope, able to think a little bit like engineers do.

\sphinxAtStartPar
:TODO:  ask Engineers without Borders or the RAE for help sourcing some nice diverse pictures.


\chapter{About this programme}
\label{\detokenize{about:about-this-programme}}\label{\detokenize{about::doc}}
\sphinxAtStartPar
This programme has been developed based on experiences within HeatHack, a group of community volunteers and students in Edinburgh that have been helping churches understand how to heat their buildings better.  HeatHack has previously been funded by the John Templeton Foundation via Scientists in Congregations Scotland, with small contributions from the University of Edinburgh’s Schools of Informatics and Engineering, and the University of St Andrew’s School of Computer Science.

\sphinxAtStartPar
It has been developed and is being run in conjunction with The Surefoot Effect, a Community Interest Company that helps communities, businesses and governments put sustainability and resilience at the heart of what they do.

\sphinxAtStartPar
The programme would not be possible without the generous support of the Royal Academy of Engineering through their Ingenious public engagement programme.

\sphinxAtStartPar
:TODO: logo and improve text

\noindent{\sphinxincludegraphics[width=200\sphinxpxdimen]{{surefoot-logo}.png}\hspace*{\fill}}







\renewcommand{\indexname}{Index}
\printindex
\end{document}